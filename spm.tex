\documentclass[a4paper,12pt]{article}
\usepackage[margin=1.00in]{geometry}
\usepackage{hyperref}
\usepackage{xurl}
\usepackage{makeidx}
\usepackage{tabularx}
\usepackage{amsmath}
% Required package
\usepackage{amssymb}
\usepackage{graphicx}
\usepackage{ragged2e}
\usepackage{algorithm}
\usepackage[table]{xcolor}
\usepackage{algpseudocode}
\usepackage{tabularx}
% Remove the red boxes around links
\hypersetup{
    colorlinks=true,
    linkcolor=black, % You can change link colors to your preference
    citecolor=blue, % You can change citation colors to your preference
    urlcolor=black % You can change URL colors to your preference
}
\begin{document}
\begin{titlepage}
   \begin{center}
        \vspace*{-8ex}
        \begin{figure}[h!]
  \raggedleft
  \includegraphics[width=15cm, height=7cm]{logo.png} 
\end{figure}
       \centering{\large Masters of Applied Computer Science}\\[0.3in]
       \centering{\large SOEN 6841: Software Project Management}\\[0.3in]
\vspace{1.0cm}
       \textbf{\large Topic Analysis and Synthesis Report} \\ [0.3in]
        \centering{\large 38. How can I involve my team in project management activities without increasing overhead?} \\ [0.3in]
    

\vspace{3.0cm}
    
         {\centering Submitted to: Pankaj Kamthan\\[0.4in]}

\vspace{2.0cm}
    
        \centering Submitted by:\\[0.15in]
        {\centering Kajal Sehrawat\\}
        {\centering 40230025\\}
vspace{1.0cm}
        {\centering (November 30, 2023)\\}
     
    \centering{Git hub:\\\url{https://github.com/kajal23sehrawat/Soen-spm/}}\\[0.5in]
        
       \vfill
   \end{center}
\end{titlepage}

\renewcommand{\thesection}{\arabic{section}}
\ TABLE OF CONTENTS

\newpage
\section*{Abstract}
\addcontentsline{toc}{section}{Abstract}
The article focuses on the positive impact of complaints on bosses and organizations, underscoring the beneficial aspects of welcoming and addressing grievances within a workplace environment. The complaint here is a trust-based mechanism of highlighting the secret issues, and giving them implicit message feedback. Moreover, complaints also show what value an individual has and that helps managers to line up their leadership with team objectives. Complaints are highlighted as a vital means of addressing conflicts, which is presented as a coaching opportunity to help individuals on the teams develop empowerment traits. A high level of conflict management not only directly impacts team cohesion but also alters the negative and positive effects of relationship conflict and task conflict, respectively, on team cohesion.
\\\\
Complaints, as described above are instrumental in giving comprehensive pictures of both short-term problems and long-term barriers and as such, serve to widen management's point of view. The study shows enhanced positive emotions and the long-term effect of leadership behaviors on employee well-being by suggesting that employees' emotional regulation, when supported by transformational leaders, reduces the negative effects on job satisfaction and stress. In summary, the findings provide useful information about present and future problems thus expanding the managerial vision. Complaining can help us to build empathy, and social connection with others, and lead to positive change.
\newpage
\section{INTRODUCTION}


\subsection{MOTIVATION}
The active participation of all the team members plays a crucial role in timely completion of the project. It is the responsibility of the team manager to make sure that all the members of the team are involved in project management activities. But doing without increasing the overhead is a challenge but is needed to enhance the project management methods and make the team more effective and efficient.
The motivation behind it is to increase efficiency by maximizing the team involvement and improve productivity within limited resources in project management processes. 




\subsection{Problem Statement}
The report puts light on the issues faced by project managers in their efforts to ensure the engagement of team members in project management activities while avoiding unnecessary overhead. Balancing the active team participation,  along with time, money and resources constraints, is demanding. This report suggests some strategies to optimize team involvement while minimizing the overhead.
  


\subsection{Objective}
The main objective of this report is to find efficient ways to increase team members' involvement in project management activities without raising the cost. This would not only prove beneficial for the organization, by optimizing their resources utilization, but also for the team members in uplifting their skills and boosting their motivation and self esteem. The report focuses on various strategies that can be opted by the project manager to enhance the team involvement in the project activities, balancing the overhead at the same time.




\newpage
\section{BACKGROUND}


\subsection{Project management Overhead}
Project management overhead means all the indirect extra cost that a company incurs indirectly while managing project-related activities beyond the core project tasks. For instance, meetings, decision-making, risk management strategies, inefficient communication between team members can escalate the project overhead. 

\vspace{1.0cm}

[1] Overhead Rate = Overhead Costs / Sales
\vspace{1.0cm}


By opting several methods and strategies, managers can minimize inefficiencies causing overhead.


\pagebreak
\section{METHODS \& METHODOLOGY }

\subsection{Simplifying the process }
\begin{itemize}
    \item Effective team
    \\Project team has a huge impact on the success of the project. While building the team, the manager should keep in mind the balance between skills, expertise and experience. Project managers should take practical decisions while assigning roles to team members ensuring that involvement of each team member is according to the requirement, their skills, and need of the project. For that purpose the team should have core members to handle critical tasks that need expertise in each phase, whereas the rest of the members have general skills.  This reduces the confusion and gives team members clarity around expectations along with feelings of ownership.
        \begin{figure}[h!]
      \raggedleft
      \includegraphics[width=15cm, height=7cm]{2.jpg} 
    \end{figure}

    \item Managing team meetings effectively
   \\It should be ensured that the project meetings are productive and focused. A well-organized meeting is the best way to keep projects on track. Whereas, unnecessary meeting scheduling is a waste of time and money. It immensely increases the overhead. To manage that, following things should be kept in mind:
   \begin{itemize}%[leftmargin=1em]
  \renewcommand{\labelitemi}{$\Rightarrow$}
 \item Schedule meetings only if it is necessary
 \item Keep the meetings simple, brief, productive and topic-oriented
 \item Include the right attendees
 \item Have well planned meeting with proper organization
 \item Keep it time bounded, start and finish on time
\end{itemize}
[3]

\item Task scheduling and prioritizing
\\ Scheduling the tasks increases productivity and helps to keep track of activity and time. It is an efficient way to prioritize the tasks and complete them on time. [4]Eisenhower Matrix (Urgent-Important Matrix) can be used to prioritize tasks by urgency and importance. It sorts out less urgent tasks.

\item External help
\\ If it is necessary then taking expertise help on contract-basis outside the team or organization can be an efficient way to simplify the process of management activities. Using external software, tools for project tracking can contribute to proper execution of these activities. Additionally, seeking guidance and perspective from external experienced advisors is good for legal matters or decision-making.


\item External help
\\ If it is necessary then taking expertise help on contract-basis outside the team or organization can be an efficient way to simplify the process of management activities. Using external software, tools for project tracking can contribute to proper execution of these activities. Additionally, seeking guidance and perspective from external experienced advisors is good for legal matters or decision-making.
\end{itemize}


\subsection{Increasing efficiency}

\begin{itemize}
    \item Control tools can be used to focus on performance levels, costs, and time schedules. 
\\ They can be very useful to:

\begin{enumerate}
\item Track the progress
\item Implement necessary corrective steps if anything seems off-track
\end{enumerate}


 \item Decision-making 
\\Decisions can impact the overall efficiency of the project. Therefore, few approaches while taking a decision to reach optimal efficiency are:
\begin{enumerate}
\item Command : Taking without help or inclusion of others.
\item Consult : Gaining insights from others before making a decision.
\item Vote : Its a quick and effective method
\item Consensus : everyone puts their views forward until a final decision is made.
\end{enumerate}
A decisive project manager can easily minimize the overhead and maximize productivity. [5]


\item Meetings
\\ Team meetings play a vital role in increasing the efficiency of any project. But sometimes they can be a waste of time for the members, hence rising overhead.
Trying to keep online meetings whenever possible instead of in-person meetings can save a lot of time. Regular meetings can be replaced by specialized meetings that focus on high-priority activities. Encourage teammates to do pre-work before each meeting.

\item Reviews and communication
\\Communication is the key. Ineffective communication can lead to misunderstanding in respect of construction projects.
\\
\\  Tubbs and Moss says that formal communication flows in 4 directions:

\begin{itemize}%[leftmargin=1em]
  \renewcommand{\labelitemi}{$\Rightarrow$}
 \item Downward communication: to provide information on goals to subordinates .
 \item Upward communication: informing upper levels about the status of lower levels.
 \item Horizontal/lateral communication: communication between people of the same level of the assigned sub-team, to improve their work and make it more productive.
 \item Diagonal communication: interaction between people of different levels aims to improve coordination.
 
\end{itemize}

\begin{figure}[h!]
      \raggedleft
      \includegraphics[width=15cm, height=7cm]{66.jpg} 
    \end{figure}



Communication is very important for reviewing and critiquing project artifacts. It provides everyone with eachother’s insights, hence making the project artifacts more effective. It also boosts the self-esteem, confidence and motivation of team members and shows their active participation in the project management process. It also has a big role in finding the loopholes in the project and adjusting and correcting them on time making the overall project more efficient.

\item Apart from these methods few other strategies can be opted for an efficient management process. Conducting training sessions for the members who are new to the domain. It will help to train the team with the concepts used in the project management and give them a practical setting for effective planning, hence increasing their overall contribution.
\end{itemize}

\subsection{Maintaining efficiency}

\begin{itemize}


\item Online Repositories
\\To streamline the processes, use VCS and centralized digital repositories like data dictionaries, aws, box. It's time efficient, convenient, and accurate.

\item Surveys 
\\Use of automated, pre-populated forms can save time, reduce errors, ensure accuracy and consistency can be done for information collection and taking reviews. It stores data in a database making it easily accessible and efficient.

\item Using platforms like MS teams or zoom for meeting makes it easier, less-time consuming and convenient for all the members.

\end{itemize}


\section {TECHNIQUE USED}

\subsection{RACI Model}
TRACI Matrix Model is widely used to get clarity regarding roles of everyone involved in a project. It is a type of responsibility assignment matrix (RAM). It is a matrix table that defines the clear roles and responsibilities, and level of involvement of individual team members which is denoted as R (Responsible), A (Accountable), C (Consulted) or I (Informed).


\begin{figure}[h!]
      \raggedleft
      \includegraphics[width=15cm, height=7cm]{What-is-a-RACI-Matrix_2.jpg} 
    \end{figure}
\\
Process of making RACI chart:
[8]
\begin{enumerate}
\item Identify work processes and list them on the left side of the chart.
\item Determine the decisions and activities to chart.
\item Identify all project stakeholders and list their roles and tasks.
\item Fill the RACI chart cells . First assign R’s then determine who has the A, followed by C’s and I’s. Every task must have at least one R and utmost one A.
\item Present the RACI chart to all the members and finalize it.
\item Keep updating the chart.
\end{enumerate}





\begin{figure}[h!]
      \raggedleft
      \includegraphics[width=15cm, height=7cm]{11.jpg} 
    \end{figure}
\\


\begin{figure}[h!]
      \raggedleft
      \includegraphics[width=15cm, height=7cm]{2.1_RACI_Matrix_Chart.jpg} 
    \end{figure}
\\


\subsubsection {Advantages:} 
\begin{itemize}
\item The RACI matrix is easy to make in Microsoft Excel or Google sheets,  and understood without any technical knowledge.
\item It clearly defines the responsibilities within the team keeping everything transparent between the team members, decreasing the misunderstandings within the team.
\item It streamlines communication and decision-making, leading to higher efficiency and reducing team confusion.
\item It saves time in a meeting
\item It provides a structured framework for management activities, making the process efficient and easy.    [11]
\end{itemize}


\subsubsection {Disadvantages:}
\begin{itemize}
\item It requires high accuracy and it’s a complex task to maintain this chart for big projects, and is time consuming initially. It needs to be updated throughout the project cycle and can sometimes be a waste of time.
\end{itemize}



\newpage
\section{RESULTS AND CONCLUSION}
From this report it can be concluded that if the project managers opt few strategies then they can enhance the involvement of their teams in project management activities without raising the overhead. Simple strategies like good communication, efficient team meetings, task scheduling and prioritizing, efficient decision making, and using controls can prove to be of great help.
\\
Another tool highlighted in the report is RACI matrix, which helps in achieving most of these strategies as it clarifies the roles and expectations from every team member in the initial stage. It also ensures that every member gets the required training for their assigned roles.




\newpage
\section{Acknowledgements}
I want to express my sincere appreciation to Perplexity and ChatGPT for providing me with good points and sources. it has enchanced the quality of my work. IEEE research papers helped me a lot. A book "5-Phase project management by JosephW. Weiss proved to be of great help. Overleaf was an online platform I used for Latex.


\newpage
\addcontentsline{toc}{section}{References}
\vspace*{-35pt}
\renewcommand{\refname}{References}
\begin{thebibliography}{99}

\bibitem{why_complaints}
 \url{https://www.adeaca.com/blog/faq-items/what-is-project-overhead/}


\bibitem{g}
\url{https://images.ctfassets.net/rvt0uslu5yqp/3CEn4if6fBS8n1zFm9LjTA/8b4f6a00350ea5d560c7351cdc6722a9/99414506-E076-4FE8-8F87-1760B5624AE3.jpeg?fm=webp&w=1920&q=75}

 \bibitem{why_complaints}
 \url{https://www.planview.com/resources/guide/what-is-project-management/project-meeting/}

 \bibitem{why_complaints}
 \url{hhttps://www.eisenhower.me/eisenhower-matrix/}

 \bibitem{why_complaints}
 \url{https://www.indeed.com/career-advice/career-development/decision--making}

 \bibitem{why_complaints}
 \url{https://haiilo.com/wp-content/uploads/2022/05/effective-communication-overview.webp}


\bibitem{why_complaints}
 \url{https://www.aihr.com/wp-content/uploads/What-is-a-RACI-Matrix_2.png}

 
\bibitem{why_complaints}
 \url{https://www.forbes.com/advisor/wp-content/uploads/2021/05/2.1_RACI_Matrix_Chart.png}


\bibitem{why_complaints}
\url{files.com/62fcfcf2e1a4c21ed18b80e6/64c2a577c3c52fbc25a7912d_when_to_use_a_raci_chart_tggk.png}


\bibitem{why_complaints}
 \url{https://project-management.com/understanding-responsibility-assignment-matrix-raci-matrix/}






\bibitem{feedback_importance}
Suhanda, R., & Pratami, D. (2021, July 31). RACI Matrix Design for Managing Stakeholders in Project Case Study of PT. XYZ. International Journal of Innovation in Enterprise System, 5(02), 122-133. \url{https://doi.org/https://doi.org/10.25124/ijies.v5i02.134}

\bibitem{}
Joseph W. Weiss, Robert K.Wysocki, 5-phase project management, A practical planning and implementation guide.\url{https://ds.amu.edu.et/xmlui/bitstream/handle/123456789/10108/5-phase%20project%20management.pdf?sequence=1&isAllo}


\end{thebibliography}

\end{document}


